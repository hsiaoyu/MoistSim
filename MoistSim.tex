\documentclass[anonymous,timestamp,review,acmtog]{acmart}

\usepackage{booktabs} % For formal tables
%\usepackage{subfigure}
\usepackage{subcaption}
\usepackage{graphicx}
%\graphicspath{{C:/Users/black/OneDrive/Documents/UT/Research/Shell/MoistInducedSim}}

\newcommand{\ba}{\mathbf{a}}
\newcommand{\bb}{\mathbf{b}}
\newcommand{\bc}{\mathbf{c}}
\newcommand{\bg}{\mathbf{g}}
\newcommand{\br}{\mathbf{r}}
\DeclareMathOperator{\tr}{tr}
\newcommand{\hn}{\hat{\mathbf{n}}}

\acmPrice{15.00}

% The next six lines come directly from the completed rights form.
% You MUST replace them with the lines specific to your accepted work.
\copyrightyear{2018}
\acmYear{2018}
\setcopyright{rightsretained}
\acmConference{Conference Name}{Conference Date and Year}{Conference Location}
\acmDOI{10.1145/8888888.7777777}
\acmISBN{978-1-4503-1234-5/17/07}

% Use the "authoryear" citation style, and make sure citations are in [square brackets].
\citestyle{acmauthoryear}
\setcitestyle{square}

% A useful command for controlling the number of authors per row.
% The default value of "authorsperrow" is 2.
\settopmatter{authorsperrow=2}

% end of preamble.

\begin{document}

% Title. 
% If your title is long, consider \title[short title]{full title} - "short title" will be used for running heads.
\title{Physical Simulation of Moisture Induced Thin Shell Deformation}

% Authors.
\author{Hsiao-yu Chen}
\affiliation{%
  \institution{University of Texas at Austin}}

\author{Etienne Vouga}
\affiliation{%
  \institution{University of Texas at Austin}}

% This command defines the author string for running heads.
%\renewcommand{\shortauthors}{DeJohnette, Rowland-Smith, Badeeri, and Foyt}

% abstract
\begin{abstract}

\end{abstract}

%CCS
\begin{CCSXML}
%<ccs2012>
%<concept>
%<concept_id>10010147.10010371.10010372</concept_id>
%<concept_desc>Computing methodologies~Rendering</concept_desc>
%<concept_significance>500</concept_significance>
%</concept>
%<concept>
%<concept_id>10010147.10010371.10010372.10010374</concept_id>
%<concept_desc>Computing methodologies~Ray tracing</concept_desc>
%<concept_significance>500</concept_significance>
%</concept>
%</ccs2012>
\end{CCSXML}

%\ccsdesc[500]{Computing methodologies~Rendering}
%\ccsdesc[500]{Computing methodologies~Ray tracing}
%
%%keywords
%\keywords{ray tracing, global illumination, octrees, quadtrees}
%
%% A "teaser" figure, centered below the title and authors and above the body of the work.
%\begin{teaserfigure}
%  \centering
%  \includegraphics[width=6.0in]{aaafiles/fountain}
%  \caption{Drumheller Fountain, The University of Washington, Seattle WA.}
%\end{teaserfigure}
%
%% Processes all of the front-end information and starts the body of the work.
\maketitle

\section{Introduction}


\paragraph{Contribution} We present a low-order discrete shell model tailored to simulating non-unform, anisotropic, differential swelling and shrinking of thin shells. In contrast to previous methods for simulating related phenomena, such as burning and growth, our formulation builds on discrete geometric shell theory and supports arbitrary rest curvature and strain, and physical settings such as thickness and Lam\'{e} parameters. We couple our shell model to a simple formulation of moisture diffusion in both the lateral and thickness directions, which takes into account anisotropy of the material grain. In a series of experiments, we show that our model successfully predicts the qualitative behavior of thin shells undergoing complex, dynamic deformations due to swelling or shrinking, such as occurs when paper is moistened, leaves dry in the sun, or thin plastic melts.

\subsection{Related Work}

in-plane growth with Loop subdivision~\cite{Vetter13}

\section{Continuous Formulation}
Before describing our discretization of shells, we briefly review the formulation in the continuous setting, as this formulation will guide our discretization. 

\paragraph{Shell Geometry} We can represent shells $S\subset \mathbb{R}^3$ of thickness $h$ by a parameter domain $\Omega$ in the plane and an embedding $\phi: \Omega\times [-h/2, h/2]\to\mathbb{R}^3$ with $S$ the image of $\phi$ (see Figure~\ref{fig:XXX}). We adopt the common \emph{Kirchhoff-Love} assumption that the shell does not undergo any transverse shear; ie, that the shell volume is foliated by normal offsets of the shell's \emph{midsurface} $\br:\Omega\to\mathbb{R}^3$. In other words,
$$\phi(x,y,z) = \br(x,y) + z\hn(x,y)$$
where $\hn = (\br_x \times \br_y)/\|\br_x \times \br_y\|$ is the midsurface normal. The shell's deformation is thus completely determined by the deformation of the midsurface. The metric $\bg$ on the slab $\Omega \times [-h/2,h/2]$, pulled back from $\mathbb{R}^3$, can be expressed in terms of the geometry of the midsurface:
$$\bg = \left[\begin{array}{cc}\ba - 2z\bb + z^2 \bc & 0\\0 & 1\end{array}\right],$$
where
$$\ba = d\br^Td\br \quad \bb = -d\br^Td\hn \quad \bc = d\hn^Td\hn$$
are the classical first, second, and third fundamental forms of the surface $\br$.

Oftentimes, the parameterization domain of a thin shell is assumed to be also the rest state of the shell, so that the strain in the material of the shell can be determined directly from looking at $\bg$. We cannot assume this: consider for instance a piece of paper whose center has been moistened by spilled coffee. The fibers in the coffee stain stretch; since they are confined by the surrounding non-wet region of the paper, the paper cannot globally stretch in such a way that both the wet and dry regions of the paper are simultaneously at rest. Instead, the paper will \emph{buckle} out of plane, into a shape that compromises between relaxing the in-plane (stretching) strain and the introduced bending strain. At this point the paper's rest state is \emph{non-Euclidean}---it is impossible to find any embedding of the paper into $\mathbb{R}^3$ that is entirely strain-free.

We therefore record the rest state of the shell using a \emph{rest metric} $\bar\bg(x,y,z).$\footnote{Here and throughout the paper, we use an overbar to denote quantities associated to the shell rest state.} Since our model is tailored to simulating differential in-plane swelling or shrinking across the thickness of the shell, we make the simplifying assumption that this rest metric is linear in the thickness direction, and does not affect the shell thickness:
$$\bar\bg(x,y,z) = \bar\ba(x,y) - 2z\bar\bb(x,y).$$
A shell that begins a simulation at rest will simply have $\bar\ba = \ba$ and $\bar\bb = \bb$; similarly, in the case that the shell \emph{does} have a rest state $\bar\br$ that is isometrically embeddable in $\mathbb{R}^3$, $\ba$ and $\bb$ are the first and second fundamental forms of the surface $\bar\br$. Therefore $\ba$ and $\bb$ can be thought of as representing the ``rest metric'' and ``rest curvature'' of the shell, respectively.\footnote{We stress, though, that these labels are to provide intuition only---$\bar\ba$ and $\bar\bb$ must not, and generally will not, satisfy usual relationships from differential geometry such as the Gauss-Codazzi-Mainardi equations.}

Finally, we cannot assume that the shell has uniform density, since different parts of the shell might gain or lose mass due to absorbing or releasing moisture. We therefore allow the density per unit \emph{rest} volume $\rho(x,y)$ to vary over $\Omega$. (In principle, we could also model the variation in density across the shell thickness; however doing so leads only to a small ($O(h^3)$) correction to the shell's kinetic energy, and since the swelling phenomena we are interested in simulating tend to happen over relatively long time scales, there is no need for such accuracy.)

To summarize, our parameterization of thin shells involves the following kinematic elements:
\begin{itemize}
\item a thickness $h$ and parameterization domain $\Omega\subset\mathbb{R}^2$, both of which are fixed over the course of the simulation;
\item an embeddding $\br:\Omega\to\mathbb{R}^3$ representing the shell midsurface's ``current''/``deformed'' geometry, and which evolves over time. From this midsurface embedding, the embedding of the full shell volume $\phi$, and the midsurface fundamental forms, can be calculated;
\item a rest metric and density, parameterized by the pair of tensor field $\bar\ba, \bar\bb$ and a scalar field $\rho$ over $\Omega$, respectively. These might also evolve over time, due to changes in the shell rest state via growth or shrinkage.
\end{itemize}

\subsection{Shell Dynamics}
Motivated by the common observation that a sufficiently thin shell bends much more readily than it will stretch, we assume that the shell's deformation involves \emph{large rotations} but only small in-plane strain of the midsurface: $\|\bar\ba^{-1}\ba-I\|_{\infty} < h.$ We also assume that the shell's material is uniform and isotropic. The simplest constitutive law consistent with these assumptions is to use a St. Venant-Kirchhoff material model\footnote{The neo-Hookean material model is also popular in computer graphics and could be used instead, although there is little benefit to doing so when simulating thin shells since strains are typically small.} together with Green strain; it can be shown~\cite{} that these choices yield an elastic energy density that can be approximated up to $O(h^4)$ by
$$W(x,y) = \left(\frac{h}{4} \|\bar\ba^{-1}\ba - I\|^2_{SV} + \frac{h^3}{12}\|\bar\ba^{-1}(\bb-\bar\bb)\|^2_{SV}\right)\sqrt{\det \bar\ba}$$
where $\|\|_{SV}$ is the ``St. Venant-Kirchhoff norm''\cite{}
$$\|M\|_{SV} = \frac{\alpha}{2}\tr^2 M + \beta \tr\left(M^2\right),$$
for Lam\'e parameters $\alpha, \beta$. In terms of the Young's modulus $E$ and Poisson's ratio $\nu$,
$$\alpha = \frac{E\nu}{(1+\nu)(1-2\nu)}, \quad \beta = \frac{E}{2(1+\nu)}.$$

We thus have a formulation of kinetic energy and potential energy
$$T[\dot\br] = \int_{\Omega} h\rho\|\dot\br\|^2 \sqrt{\det\bar\ba}\,dxdy, \quad V[\br] = \int_{\Omega} W(x,y)\,dxdy,$$
to which additional external energies and forces (gravity, constraint forces, etc) can be added to yield equations of motion via the usual principle of least action.

\section{Discretization}

\section{Moisture Diffusion}

\section{Result}
\subsubsection{Radially Wet Disc}
\begin{figure}[h]

\begin{subfigure}{0.5\textwidth}
\centering
\includegraphics[width=\linewidth]{WetRimSim1.png} 
\caption{Caption1}
\label{fig:subim1}
\end{subfigure}%
\hfill
\begin{subfigure}{0.5\textwidth}
\centering
\includegraphics[width=\linewidth]{WetCenterSim2.png}
\caption{Caption 2}
\label{fig:WetDisc}
\end{subfigure}%

\end{figure}

\subsubsection{Machine Direction Experiment}
\begin{figure}[h]

\begin{subfigure}{0.45\textwidth}
\centering
\includegraphics[width=\linewidth]{RechSim1.png} 
\caption{Caption1}
\label{fig:subim1}
\end{subfigure}%
\hfill
\begin{subfigure}{0.45\textwidth}
\centering
\includegraphics[width=\linewidth]{RecvSim1.png}
\caption{Caption 2}
\label{fig:Rec}
\end{subfigure}%

\end{figure}

\subsubsection{Dried Leaves}
\begin{figure}[h]
 
\begin{subfigure}{0.45\textwidth}
\centering
\includegraphics[width=0.45\linewidth]{actualLeafH2.jpg} 
\caption{Caption1}
\label{fig:subim1}
\end{subfigure}%
\hfill
\begin{subfigure}{0.45\textwidth}
\centering
\includegraphics[width=0.45\linewidth]{actualLeafV.jpg}
\caption{Caption 2}
\label{fig:DriedLeaf}
\end{subfigure}%

\bigskip

\begin{subfigure}{0.45\textwidth}
\centering
\includegraphics[width=0.45\linewidth]{LeafhSim1.png} 
\caption{Caption1}
\label{fig:subim3}
\end{subfigure}%
\hfill
\begin{subfigure}{0.45\textwidth}
\centering
\includegraphics[width=0.45\linewidth]{LeafvSim1.png}
\caption{Caption 2}
\label{fig:subim4}
\end{subfigure}%
 
\caption{Simulation for Dried Leaves}
\label{fig:image2}
\end{figure}
\subsubsection{Wetting Paper Annulus}
\begin{figure}[h]
 
\begin{subfigure}{0.45\textwidth}
\centering
\includegraphics[width=0.45\linewidth]{AnnulusSim1.png} 
\caption{Caption1}
\label{fig:subim1}
\end{subfigure}%
\hfill
\begin{subfigure}{0.45\textwidth}
\centering
\includegraphics[width=0.45\linewidth]{AnnulusSim2.png}
\caption{Caption 2}
\label{fig:DriedLeaf}
\end{subfigure}%

\bigskip

\begin{subfigure}{0.45\textwidth}
\centering
\includegraphics[width=0.45\linewidth]{AnnulusSim3.png} 
\caption{Caption1}
\label{fig:subim3}
\end{subfigure}%
 
\caption{Simulation for Wet Annulus}
\label{fig:image2}
\end{figure}

\subsubsection{Wetting Straw Wrapping Paper}
\subsubsection{Melting Spoon}
\subsubsection{Melting Torus}
\begin{figure}[h]
 
\begin{subfigure}{0.45\textwidth}
\centering
\includegraphics[width=0.45\linewidth]{TorusSim1.png} 
\caption{Caption1}
\label{fig:Torus1}
\end{subfigure}%
\hfill
\begin{subfigure}{0.45\textwidth}
\centering
\includegraphics[width=0.45\linewidth]{TorusSim2.png}
\caption{Caption 2}
\label{fig:Torus2}
\end{subfigure}%

\bigskip

\begin{subfigure}{0.45\textwidth}
\centering
\includegraphics[width=0.45\linewidth]{TorusSim3.png} 
\caption{Caption1}
\label{fig:Torus3}
\end{subfigure}%
 
\caption{Simulation for Dried Torus}
\label{fig:Torus}
\end{figure}

\section{Conclusions and Future Work}

%Nullam vulputate enim ut tortor mollis pharetra. Cras pellentesque sem a accumsan malesuada. Donec at massa nisl. Sed malesuada felis id nisl maximus efficitur. In pretium metus non faucibus pulvinar. Sed pulvinar elit ultrices mauris vehicula, id ultricies purus finibus. Fusce tempus elit molestie, consequat ipsum eget, iaculis nibh. Cras tincidunt, orci in lacinia tempus, mauris leo finibus orci, vitae dignissim dui risus et odio. Sed commodo ultricies nulla, et varius velit aliquam quis. Sed efficitur, ex non facilisis dignissim, lacus orci accumsan massa, dictum facilisis arcu lacus ac leo. Sed quis tellus dictum massa egestas dapibus vel et justo. Nulla euismod lectus ut purus hendrerit porttitor. Suspendisse quis dui ligula. Proin non porta libero. Maecenas vel feugiat urna.
%
%\begin{acks}
%The authors would like to thank Dr. Yuhua Li for providing the MATLAB code of the \textit{BEPS} method.
%The authors would also like to thank the anonymous referees for their valuable comments and helpful suggestions. The work is supported by the \grantsponsor{GS501100001809}{National Natural Science Foundation of China}{https://doi.org/10.13039/501100001809} under Grant No.: ~\grantnum{GS501100001809}{61273304}
%21 and ~\grantnum[http://www.nnsf.cn/youngscientists]{GS501100001809}{Young Scientists' Support Program}.
%\end{acks}

\bibliographystyle{ACM-Reference-Format}
\bibliography{MoistSim}
\end{document}
